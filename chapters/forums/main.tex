\chapter{Fóruns}
\label{cap:forums}

\begin{quotation}[]{Yoko Ono}
The computer is my favourite invention. I feel lucky to be part of the global village. I don't mean to brag, but I'm so fast with technology. People think it all seems too much, but we'll get used to it. I'm sure it all seemed too much when we were learning to walk.
\end{quotation}

De acordo com o dicionário, fóruns são um lugar ou meio onde ideias e visões sobre um particular assunto podem ser trocadas.

\section{Histórico}
Um dos primeiros fóruns de perguntas e respostas a surgir foi o Answer Point, disponibilizado pelo Ask Jeeves na década de 90. Seu slogan era: "Answer Point é o lugar onde você pode fazer e responder perguntas. Tem uma dúvida? Publique-a! Sabe a resposta? Publique-a!" \cite{websearchhistory}.

 De acordo com Jim Lanzone, vice-presidente sênior do AskJeeves, a maior dificuldade era pra incentivar usuários à responder. Com poucas questões respondidas, Jim Lanzone defende a utilidade dos engenhos de busca, pois a maioria das buscas são únicas, ainda que a relevância não seja perfeita, é possível trazer resultados amplos \cite{websearchhistory}. Além disso, esperar por uma resposta é conflitante com o que o usuário mais precisa, velocidade. 

A Ask Jeeves possuia outros produtos além do AnswerPoint, o seu principal era o Ask.com, um buscador concorrente ao do Google. Segundo Jim Lanzone, a empresa decidiu focar em outros aspectos do seu buscador e descontinou o projeto do AnswerPoint \cite{websearchhistory}.

Na mesma semana do fechamento do Answer Point, foi lançado o Google Answers \cite{googleanswerswiki}. O Google Answers era um serviço que permitia que as pessoas submetessem perguntas e oferecessem um pagamento por sua resposta, este pagamento podia variar de 2,50 dólares até 200.50 dólares. Entretanto, também era possível ver respostas para outras perguntas já respondidas sem efetuar pagamento. Seu fracasso, dentre outros motivos, foi decorrente principalmente do surgimento de serviços de perguntas e respostas gratuitos.

O Yahoo Respostas foi criado pela empresa Yahoo no ano 2005. Foi um dos primeiros a implementar o crowdsourcing \footnote{Crowdsourcing é um modelo de produção que conta com a força do coletivo para desenvolver soluções.} com sucesso. A empresa implementou um sistemas de pontos para incentivar os usuários à responder perguntas. Com esses pontos, os usuários possuem mais acesso à plataforma, os permitindo publicar mais questões e respostas. A pontuação é atribuída pelo autor da pergunta e também pela comunidade que pode manifestar sua opinião \cite{yahooanswerswiki}.

Em 2004, Tim O'Reilly cunhou o termo de Web 2.0, ele se referia à uma web participativa, com conteúdo produzido pelos próprios internautas \cite{webtransition}. Atualmente, com a popularização dos dispositivos móveis, facilitou-se o acesso do usuário à rede e, consequentemente, sua participação. Os usuários passaram a confiar e depender cada vez mais desses meios de interação, dentre eles, os fóruns, o que tornou necessária uma web mais organizada, chamada de Web 3.0 por John Markoff. Na Web 3.0, a informação é estruturada não somente em meios inteligíveis por humanos, mas também por máquinas. Por meio das máquinas, se torna possível um acesso mais rápido à informação. É um uso mais inteligente do conteúdo já disponibilizado online.
\section{Exemplos}
Com o crescimento exponencial da internet, houve espaço para novos fóruns crescerem e se desenvolverem em nichos diferentes. São estes alguns exemplos relevantes.
\begin{itemize}
    \item \textbf{Quora\footnote{https://quora.com/}}. É um espaço pra perguntas e respostas de qualquer domínio. A empresa foi criada por Adam D'Angelo e Charlie Cheever e disponibilizada ao público em 2010.

    O fórum Quora tornou-se uma opção mais usada pelos usuários devido à um uso mais consciente dos fóruns, onde perguntas devem ser mais bem elaboradas e bem estruturadas. Perguntas que não são adequadas são reportadas e removidas da plataforma.
    
    Isso tornou-se necessário pois o Yahoo Respostas estava se tornando muito poluído com perguntas vagas \cite{quoravsyahoo}, como a de estudantes para responder tarefas de casa e com respostas sem sentido, às vezes apenas com "Eu não sei", que serviam para que ainda assim ganhassem pontos na plataforma.
    
    Graças à esta nova visão, usuários da plataforma têm um acesso facilitado às perguntas de seu interesse e menos conteúdo duplicado, além de que agora a plataforma consegue melhor recomendar perguntas aos usuários que podem a responder.
    
    \item \textbf{StackOverflow}.
    Criado por Jeff Atwood e Joel Spolsky em 2008. Jeff era um desenvolvedor e possuia um blog chamado de Coding Horror\footnote{https://blog.codinghorror.com/} onde abordava temas relacionados à programação \cite{codinghorror}. No mês de julho de 2008, quando foi criado, Jeff limitou o acesso para aqueles que eram assinantes do blog e no mês de setembro tornou público o acesso.
    
    O sucesso da plataforma foi tamanho que permitiu que crescesse para um conglomerado de fóruns de perguntas e respostas sobre vários outros tópicos, conhecido com StackExchange, onde são mantidos o próprio StackOverflow, AskUbuntu, SuperUser e mais de 150 outros. Atualmente, a plataforma possui mais de 17 milhões de perguntas.
\end{itemize}
\section{Usos Mercadológicos}

\subsection{Vendas Direcionadas}
Uma das formas mais comuns de se gerar renda é através de anúncios direcionados. Um dos que usam essa abordagem é o Quora, que hoje é avaliado em mais de 1,8 bilhão de dólares \cite{quorarevenuemodel}.

\subsection{Recrutamento}
Ao responder perguntas, os usuários criam notoriedade pra si. O StackExchange usa essa vantagem tanto para estimular usuários para responder perguntas, quanto pra identificar potenciais para determinadas vagas.

Apresentadas em forma de anúncio, vagas são ofertadas para os usuários de acordo com o seu perfil na plataforma. Da mesma forma, os recrutadores tem acesso à informações mais detalhadas dos candidatos, graças à participação dos mesmos, que os garante reconhecimento na comunidade e troféus fictícios na plataforma.

%\section{Projetos}
%\subsection{Question Answering}
%\subsection{Identificação de perguntas duplicadas}
%\subsection{Recuperação de Perguntas}
\section{Sumário}